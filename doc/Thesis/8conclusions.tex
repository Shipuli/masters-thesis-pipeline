\chapter{Conclusions}
\label{chapter:conclusions}

%Time to wrap it up!  Write down the most important findings from your
%work.  Like the introduction, this chapter is not very long.  One to
%two (never over three) pages might be a good limit. Still, the chapter
%gives the background, goals, content, and the findings. However, all that
%should already be in the previous chapters. This is just a summary (as
%are the abstract and the introduction).

%For making PDF/A version requested by the Aalto Library, open the end result pdf file in Acrobat and store it as PDF/A. Then verify the result (everything should be fine, at least as PDF/A-2b version works).

In this thesis, we examined big data pipelines to analyze stock market data.
We conducted literary research on current trends on methods that are used to analyze stock market data.
From this we saw that deep learning models are currently the driving force in stock market analysis.
Other statistic methods are also used with conjuction of data from other varying sources.
These methods are not only used to predict prices in hopes for profit, but can also be used to analyze e.g causalities in other phenomenons.

We researched what are the technologies currently used in pipelines that analyze stock market data covering both academic and industrial use and saw that public information about these is quite limited.
With the information publicly available, we saw that usually only the analysis phase was reported and other aspects of the systems were dismissed.
These other aspects were most of the time monitoring of the system, but also ingestion and storage were not reported in many cases.

We did a literary study on technologies that were used in the existing pipelines and technologies that seemed promising to be used in these pipelines.
These included technologies for ingestion, storage, monitoring and analysis parts of the pipeline and we saw that the analysis part is dominated by Spark solutions at the moment of writing.
In ingestion and storage there exists a lot of alternatives but none of them dominated field because of different use cases.
For our pipeline, where we wanted the pipeline to be as flexible as possible we opted to Kafka for ingestion and HDFS for storage.

We planned and implemented a basic stock data analysis pipeline based on the results from the previous literary studies.
Our goal was to build a pipeline that would make analysis of stock market data in big data environment more accessible to data scientist who do not have that much knowledge on big data system development.
The pipeline we built is open-source that anyone can use as a base or use it as it is.

We were not able to build multiple pipelines for comparative results because of ambigious scope and problems with the technologies used, but we managed to produce a lot of valuable practical information about developing these systems.
Our pipeline did manage to fullfil most of the requirements it was given at the start of the development, but fields such as good monitoring are still lacking.

We can say that building a machine learning pipeline in a big data environment is not trivial yet.
A lot of information is scattered, missing or going out of date as newer versions and technologies are introduced.
Building a big data pipeline requires integration of multiple technologies and because of the amount of the possible integrations, there  exists pairs of technologies that are not that compatible yet.
This thesis was able to test only a couple of these possible integrations but the ones that were tested did not integrate with each other without challenges.

In the future, hopefully, more information can come available if and when these technologies gain popularity and the role of big data grows.
Until then we have to cope with the information available and just try to produce more information to make the field more accessible for new developers.
This thesis hopefully alleviates someones process, in the field of stock data analysis.
