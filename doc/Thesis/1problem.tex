\chapter{Requirements}
\label{chapter:problem}

In this chapter we examine the problems that a novice data scientist faces when starting their process to implement big data stock market pipeline.
In this chapter we will expand the problem introduced in the introduction chapter and build the requirements for the methology we will proposing in the following chapters.
We will start by defining the target group that these problems affect the most.
These problems are not universal and to understand better to whom this thesis is benefitting the most we will be defining the target audience and the reasoning why would they be interested in the subject of this thesis.
After this we will be defining the problem itself by dividing it to smaller parts and examining these parts individually turning them to the needed requirements.

\section{Novice data scientist}

As stated in the introduction, this thesis is meant for novice data scientists that have little to no experience on developing big data systems, but have general information on data analysis and want to use state of art methods to analyse stock market data.
We use the term data scientist as it is usually used when talked about a person that does data analysis with enormous amount of data.
This is in contrary to data analysist, which is in the same data analysis domain, but works with smaller amounts of data where one does not need to worry about optimization of calculations when the data is processed \cite{voulgaris}.
Because of this optimization aspect, data scientist must have knowledge not only about complex data analysis methods but also knowledge about developing and programming systems that scale.
They usually also do not have much experience in "DevOps".
The term "DevOps" can be defined as "...a set of practices intended to reduce the time between committing a change to a system and the change being placed into normal production, while ensuring high quality." \cite{devops}.
This is usually not a point of consideration when developing ad-hoc data analysis systems on a small scale but when the system is used for a longer time in much bigger scale this is becomes important \cite{devops}. 

%\begin{sidewaystable}
%\centering
%  \caption{Novice data scientist}
%  \begin{threeparttable}
%     \begin{tabular}{|p{2cm}|>{\centering\arraybackslash}p{2cm}|>{\centering\arraybackslash}p{3cm}|p{3cm}|p{4cm}|p{4cm}|}
%      \hline
%      & Data analysis & Programming & Devops & Description & Examples \\ \hline
%      Software Developer & Might know some & Knows a lot & Knows nothing - Knows a lot & Expert at building a software & Computer science graduate*, self-taught employed \\ \hline
%      Data analysist & Knows a lot & Might know some & & Expert at data analysis with reasonable amount of data ($<$ 1TB) & Mathematics graduate \\ \hline
%      Novice Data scientist & Knows a lot & Knows some & Knows nothing - Knows little & Has interest in both fields but no real in experience in computer science & Mathematics graduate, Starting computer science student* minoring statistics \\ \hline
%      Data scientist & Knows a lot & Knows a lot & Knows a lot &  Expert at both fields & Mathematics graduate minoring large-scale computing (or vice versa) with 1+ year(s) of relevant worklife experience. \\ \hline
%      \end{tabular}
%      \begin{tablenotes}\footnotesize
%        \item[1] * Graduate here and following meaning bachelor or master level graduate
%        \item[2] * Student here meaning master level student
%    \end{tablenotes}
%  \end{threeparttable}
%\end{sidewaystable}

There is still a lot of hidden potential when it comes to stock market.
With the arise of the amount of data from social medias and other new sources, there are new ways to analyse the stock data and at the same time there are new uses for the stock market data itself to understand these new sources of data.
There is a lot of potential for new innovations when it comes to this data.
Companies who want to benefit from these kind of innovations have their own highly professional teams that have resources to analyze this kind of data in large scale.
However, the situation is different for smaller startups, which might have great ideas concerning the field but no resources to implement them.
This is where the need for novice data scientists is and this is where one can find persons who benefit from this thesis the most.

As there exists a lot companies that have products to analyze stock data why would the reader then be building their own pipeline instead of using some ready-made product.
Stock market analysis is a subject that has been researched for decades now and because of this there exists already a lot ready made tools for it which those who have enough money can use \cite{metastock} \cite{worden}.
However, with the rise of modern machine learning as the de facto way to conduct stock market analysis, these tools have not grown with this and are usually only made for statisticians or have very limiting capabilities concerning machine learning.
This makes them not suitable for any data scientist to use who wants to try the latest methods and experiment ideas that have not been done before.

Other good question is that why would the reader then be interested in using methods to analyse the big data instead of using traditional methods that the data analysist would use.
As the stock market data has been researched for decades, there is already a lot of knowledge about the stock analysis using the timeseries data with statistics.
This is why the current interest is in using big data with machine learning to learn new things about this field which has been researched a lot.


\section{Inaccessibility of big data stock analysis}

Now that we have expanded our typical novice data scientist term and examined the reasons why would they be interested in the field of stock data in big data context, next we are going to examine what are the obstacles these types of people face when trying to start data analysis on stock market data using state of art methods.
These are not problems that everyone who works in the field faces or do they mean that there is something majorly wrong with the current methods.
These are more of a problems that can make it harder for new poeple in the field to get their ideas heard.
We will use these problems to define the requirements for our proposed methology in the following chapters.

The problems can be seen to stem from three main factors; the information needed is scattered, the information needed is outdated and the needed information is lacking all together.
We start by examining the obstacles in basic stock data analysis and then move to the technical side and examine obstacles in big data technologies to implement this analysis.
Finally we end this chapter with a section that using this knowledge aggregates requirements for the system.

\subsection{Obstacles in stock data analysis}

As stated before stock data analysis is a subject that has been researched a lot and there exists a lot of materials on the subject that anybody can obtain.
However, the material that is publicly available can be quite outdated compared to the state of art research.
Much of the research is can be assumed to be carried by private companies which keep their findings as market secrets which is understandable taking into account their monetary value, but this makes it hard to not invent the wheel over and over again.

The papers published by researchers give us a better look at how, for example, machine learning is used in the stock market analysis.
The problem with these papers is usually that they report their findings on a very high level usually focusing on theoretical side and keep their actual implementations private \cite{le} \cite{adresic} \cite{islam}.
So the papers do provide valuable information on the subject but usually leave out important information how to reproduce the models which is vital information especially for the target audience of this thesis.

The problem is somewhat similar in the industrial side, although where in academic papers the theoretical side of algorithms is usually well explained, in industrial public information this side is usually left out.
The information about industrial pipelines is very limited and usually the information available focuses on promoting some product that the pipeline uses instead of the pipeline itself \cite{palmer} \cite{snively}.
For novice data scientist this gives a climpse on how the industrial pipelines are implemented, but leaves a lot of details out on what are the algorithms these pipelines implement on data and how they are implemented in practice.

Availability of the actual stock data is also a somewhat of a problem.
During the rise of internet, stock market data has been publicly available through services such as Yahoo Finance and Google Finance.
However, as the price of the stock data has grown over the years both of these services have been shut down \cite{lotter}.
There exists some services which offer the same types of data, but there are still many papers quote the Yahoo Finance as their data source although it has been 2 years since it was shut down \cite{serez2} \cite{le}.
So the needed stock data can be also hard to obtain.

\subsection{Obstacles in big data frameworks}

The situation of available information is much better on the big data tooling side.
There are a lot of open-source solutions for many different use cases and these are usually very tested as they are.
The problems arise when we have to filter the right technologies for the current domain as the amount of information is large.
If one finally finds the right technologies to their use case, they are faced with a challenge to integrate and run these tools. 
In this section we are going to examine these problems.

For a novice data scientist the number of possible technologies you can use for stock data analysis can be overwhelming.
The reason for this is that as there are a lot of technologies which all have different methods of solving their problems, none of these really stand out as the de facto solution for the stock market domain.
So to choose the perfect solution for each use case can be a tremendous job as there are a lot of different factors to weight in.

This itself is not hard to overcome but what makes this really hard is that although the advantages and disadvantages of a particular technology for one specific purpose is well documented, the integration of one technology to a plethora of others is usually not.
This is due to the constant development of each technology and the vast amount of different possible integrations.
So it is really hard for a novice data scientist to pick technologies for their pipelines that seem to be the best solution individually while not leading to a deadend because the chose technologies do not work together.

Once the technologies, that seem to work with the problem instance you have, have been chosen the next step is to run them.
This is usually documented very well for quickly starting the development on a single machine \cite{kafka} \cite{flume}, but these are usually instructions that are only meant for testing a single instance setup and are far from production ready ways to run the program or even develop it efficiently.
To put this in other terms, the methods described are not sustainable in the long term.
What this usually means is that the novice user either naively starts to develop their program on top these instructions that need a lot of revisions in order to run in production or have to spend a lot of time learning the framework and its nuances in order to build themselves a future proof development setup.

In this chapter, we examined the obstacles a lot of novice data scientists can face when starting their journey on big data stock market analysis.
In the following chapters we will try to tear down these obstacles, while confirming that these problems do exist.
Our goal is not to solve every problem that is listed here, but instead alleviate some of the burdens that these can introduce to a starting novice data scientist.

\subsection{Requirements for the methology}

As stated in the introduction, the goal of the methology is to make big data analysis for stock market data more accessible to novice data scientists and as such we want it to fulfil two key goals:

\begin{itemize}
  \item Make stock big data pipeline building easier for novice data scientist using it
  \item Make stock big data pipeline building easier for novice data scientist that read about/use the pipelines built using this method
\end{itemize}

The problems we are trying to solve here are not possible to solve with one person alone.
This is why we want the methology itself to be able to alleviate these problems if it is used enough and this is why we define the second goal here as it can help us define more meaningful requirements.

As we saw in the previous two sections there are a lot problems that can hinder the development process of a novice data scientist.
However, we are only going to focus only on a subset of these as we do not have the ability to solve everything although as we will see the other problems do affect our lives.
The specific problems we want to alleviate and the requirements that are derived from these can be seen in table 3.2.
To make referencing these requirements in the following parts of the thesis easier, we have labeled these and following requirements with labels R1 - R8.

\begin{table}[! htbp]\centering 
  \caption{Novice data scientist problems and requirements they impose}
  \begin{threeparttable}
      \begin{tabular}{|p{6cm}|p{6cm}|}
      \hline
      Problem & Requirement \\ \hline
      Lack of information/Outdated information on used data sources &  A way to find relevant data sources (R1)\\ \hline
      Vast amount of possible technologies &  A way to filter relevant technologies (R2)\\ \hline
      Documentation lack info on building sustainable pipelines & Provide a way to build pipelines that are sustainable (R3) \\ \hline
      Papers and industrial presentation lack practical information to rebuild the pipelines & Provide a way to build pipelines that are easy to reproduce (R4)\\ \hline
      \end{tabular}
  \end{threeparttable}
\end{table}

As the mapping from novice data scientist problems to requirements is quite trivial and we have already examined the problems in previous chapters we then move to the requirements that the underlying domain imposes.
We have gathered these requirements in the following list based on the literary research we did on the stock market analysis in the previous chapter:

\begin{itemize}
  \item Ability to add and ingest multiple data sources (e.g social media feeds) (R5)
  \item Ability to efficiently store possible big data (R6)
  \item Ability to efficiently process possible big data (R7)
  \item Ability to support deep learning models (R8)
\end{itemize}

The three first requirements are somewhat trivial for a normal big data system, but the last one is the one might propose some restrictions to our choices.
These are purely selected on based on the state of art research and the thought process on how to enable these methods in the developed pipeline.
This concludes our chapter for requirements based on the problems novice data scientists face and couple of requirements that the underlying domain imposes.