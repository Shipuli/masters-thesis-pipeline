\documentclass[article,11pt]{article}

\usepackage{fullpage}
\usepackage{url}
\usepackage[sorting=none]{biblatex}
\usepackage[english]{babel}
\usepackage[utf8]{inputenc}
\usepackage{amsmath}
\usepackage{graphicx}
\usepackage[]{algorithm2e}
\usepackage{enumitem}
\usepackage{bm, amssymb}
\usepackage[toc,page]{appendix}
\usepackage{listings}

\bibliography{sources}


\title{Master Thesis Proposal}
\begin{document}
\author{Ville Vainio}
\maketitle

\section{Context}

The modern economy revolves around stock market. 
Stock market is way for companies to obtain capital which they can invest into their own business. 
In exchange, person who invests into the companies stocks technically owns a piece of the company. 
The investor can make profit by selling these stock in a higher price or by receiving dividends from the company itself.

The price of the stock is simply determined by the law of supply and demand. 
If somebody is willing to pay a higher price for the stock then the price of the stock can grow. Because of this the stock market is in continuous fluctuation where people are selling and buying the stocks with the price they think the stock is worth using stockbrokers as the middleman. \cite{person}

There are many strategies on how to invest into these stocks which depend on multiple factors such as; how much do you expect to profit with your investment, how much are you willing to take risk, do you want to make money by selling the stocks or by receiving dividends and so on.
The underlying principle with every strategy is to minimize the risk you need to take in order to gain as much as profit as possible.
Some of the strategies are based on subjective evaluation of the companies, but more technical strategies use metrics that are calculated from the financial statistics or the real-time market values.
Strategies using the former data are called fundamental analysis and the strategies using latter data technical analysis.
Neither of these approaches can predict the future of the market, but can statistically decrease the probability of larger losses in the market for the investor altough the probability of large losses is still not zero with these methods. \cite{fox}

Fundamental analysis is based on the idea that each stock has a intrinsic value that can be larger than the actual price of the stock in the market and buying these will eventually lead to profits.\cite{sohnke}
The fundamental analysis focuses on the financial metrics that consist of companys overall statistics.
These are for example how much the company has made profit, how much the company has paid dividends and what is companys cash flow.
These tell a lot about the growth of the company and how the future of the company looks like.
These metrics are usually published quarterly four times a year and present more long-term statistics about the company.
Because of this, the amount of data these values present is quite small in terms of space.

The technical analysis that focuses on the real-time market values, on the other hand, needs new data almost daily.
Stock exchanges are usually open from morning, opening around 8 to 10am, until evening, closing around 5 to 7pm on weekdays.
Before and after this there are more limited pre- and after-hours trading which lasts usually around 1 to 2 hours depending on the exchange in which more limited stock trades can be made.
uring these hours multiple values are recorded on the prices of the stock from which the most important ones being: the highest price the stock was sold, the highest price the stock was sold and the number of stocks traded during the time interval.
The technical analysis focuses on finding recognizable patterns through this data. \cite{murphy}
Where the data used by the fundamental analysis was relatively small, these values can generate gigabytes of raw data in a week.

% for screenshots
\begin{figure}[h]
    \includegraphics[scale=0.43]{system2} 
    \centering
    \caption{Application scenario for the thesis}
\end{figure}

\section{Research Questions}
% research and engineering questions (why do you have to do it and why it is new)

Both of these analyses are valuable to investors, but implementing them on a large amount of companies is a difficult task for non-technical individual.
This is why some technology companies offer this as a paid service.
Here we present an application scenario that is based on one such product.
From this product we apply the data sources of the system, the requirements of the possible extensions and the requirements of the results to create a realistic application scenario.
This scenario will be used to throughout the thesis to create meaningful results to companies building this kind of systems.

This application scenario is presented in figure 1.
In the figure, the components market with solid lines represent the core system functions, and the dashed lines represent add-on functionalities that should be possible to extend into the system in the future.
The requirements of the system should be as follows;
The system should produce long ($r_1$) and short ($r_2$) term predictions of stock prices.
The computation of long term predictions can take time because of the nature of these predictions but the short term predictions should be between two minutes to one hour available.
Long term predictions are the result of fundamental analysis ($f_2$) and historical technical analysis ($f_4$) whereas the short term prediction should come mostly from quick technical analysis ($f_3$).
The cost of the system should grow logarithmically/linear with time meaning that the cost of processing and storing data should not exponentially increase over time.
Finally, the core system should be able to fulfill these requirements for at least the +5000 companies in the major U.S stock markets.

The data to the application is ingested from two main types of data sources quote and fundamental.
These sources consist of values that were briefly described previously in technical and fundamental analysis respectively.
The quote data is usually updated with minute intervals depending on the provider whereas the fundamental data does not change so often.
Theoretically, the fundamental data can change anytime, because of dividends which can be payed whenever the companies want but this does not happens often and this value is only used in long term fundamental analysis so longer update intervals are acceptable.
In the figure, these are separated into the U.S market ones ($d_1$ and $d_2$) and the other sources ($d_3$ and $d_4$) that provide the same data on other global markets.
The extendable global data sources are grouped into one box but in reality this data would be ingested from numberable different providers as there is no single entity at the time of writing this that provides all of this data.
Theoretically the maximum size of this extendable data would be 5Gb per day which is extrapolated in from the U.S market data based on statistics that in January 2019 there were globally 51 599 companies listed in the stock markets \cite{global}.
This amount can and will fluctuate as companies enter and exit the markets but it gives us the scale of data we are working with.

Today, stock market analysis has also a large focus on predicting stock prices using secondary data sources that can have reflect and affect the prices of stocks. 
These secondary data sources can be anything but at the moment one of the most researched sources are traditional media and social media data.
Examples of using this kind of data to predict predict stocks can be found in \cite{kao}, \cite{skuza} and \cite{wai}.
This is why the system should have the ability to extend to ingest data from arbitrary secondary sources ($f_6$ and $d_5$ in the figure) to provide more versatile predictions about the stocks.
The amount of this data can be unlimited but is restricted to relevant sources.

Data is ingested from these data sources mainly using HTTP-protocol as this is the main method that these services ($d_1$ - $d_4$) provide.
Other possible methods that are usually available are Excel sheets and sometimes websockets, of which the websockets can be actually useful in cloud system.
Here we have separated the main ingestion functions into two main types of functions $f_1$ and $f_2$. 
$f_1$ is constantly polling and processing data whereas $f_2$ handles batch processing.
Both of these function store their raw output into the storage, but $f_1$ passes this also to the immediate technical analysis.

The system has two technical analysis functions.
For methdos that allow streaming updates there is $f_3$, which can for example be cumulative/reinforced ML models and for methods that need historical data in order to calculate the prediction, there is $f_4$.
For fundamental analysis, there is no a specific function as the introduced data sources usually provide these values pre-calculated and these values are usually easy to calculate dynamically with little to none amount of processing.

Finally, at the center of the system is the storage which is used to store the calculated predictions as well as the raw data from the data sources for later analysis.
For historical technical analysis, $f_4$ the storage should provide reasonable range query times when quering historical data and for the results the storage should provide efficient point queries for the results (< 1s).

% Orchestrating data ingesting for both of these analyses is common task on every system that provides automatic calculation on the metrics that the data scientists use to calculate predictions. When stock exchanges consist of thousands companies, which is the case e.g with the american stock exchanges, the total amount of data grows enormously. Because of this, the process of ingesting the data becomes harder to implement efficiently so that the user does not notice notable delay in the calculated values. Other problems arise from scalability when this kind of system is scaled to ingest market data not just from the U.S market but from other markets too. The most crucial problem being the the cost-efficiency of the system from the business standpoint.

So as can be seen from this application scenario the ingestion and storage of stock data is not a trivial problem to solve.
Data ingestion in the context of stock data is also a problem that has not been researched that much.
When searching with terms "big data" and "stock" the resulting papers are mostly focused on the aspect of analysing the stock data and these papers mostly ignore the steps of ingesting and storing this data.
Examples of this kind of papers are \cite{wu}, \cite{aghakhani} and \cite{kao}.
The main questions this thesis would try to answer would be: What are the requirements and needs for this kinds of system in the context of stock data, based on this what are the state of art techologies to implement this kind of system efficiently in the context of the introduced application scenario and which solution would be the best cost/perfomance-wise.

\section{Expected Outcome}

For the first research question "What are the requirements and needs for this kinds of system in the context of stock data?" this thesis plans to provide an analysis of the necessary stock market data and its usages.
From this analysis, the thesis would derive the main requirements for the system to fulfill in order to satisfy the needs of the possible subsequent analysis stage.
This result could then be used in the future if one would want to build their own ingestion system from the ground up as a base.

For the second question "What are the state of art techologies to implement this kind of system efficiently?" the thesis would perform an analysis on the current trends in data ingestion solutions.
The thesis would provide information on the latest open-source technologies that could be used to implement this kind of system, how these would fulfill the requirements introduced by the first research question with application scenario and conclude this with a comparison of these technologies on what are the advantages of using one over another.
The result of this part could be used to decide what seems to be the most suitable technology to use to implement the application scenario technically. 

For the last question "Which solution would be the best cost/perfomance-wise?" the thesis would implement open-source prototype solution based on the results of the second research question.
This prototype could be used by anybody (company or individual) as it is or as a base to build a more complex system on top of it.
The system would be targeted to practically implement the described application scenario, which could be used by companies with similar products allowing them to write more complex analysis algorithms based on the larger amount of data.

\section{Approach}

\subsection*{Step 1}

The thesis would start by going through scientific papers about stock markets and stock analysis.
There would be first text generally about the characterics of the stock markets, what are they based on, what affects them, can they actually be predicted (random walk hypothesis) and so on.
Then the thesis would go through the main directions of analysing the stock markets (fundamental analysis, technical analysis) and explain briefly some of the methods (about three methods per direction) that characterize these directions (Gordon model, Magic formula, LSTM etc.) focusing on the data that these methods need in order to calculate their predictions.
This part would be concluded by deriving the requirements for the system based partly on these analysis methods and their data needs, and partly on the definitions that make up a scalable, secure and stable cloud system.

\subsection*{Step 2}

The next step would be to perform literary research on what is currently used to perform big-data ingestion and storing, selecting from the list of technologies mostly those that seem to fulfill the requirements derived in the step 1 and would fit in the application scenario introduced in figure 1, specifically in $f_1$ and $f_2$.
For data ingestion, these technologies would probably be Apache NIFI, Apache Flume, Fluentd etc. 
For data storage, these could be HDFS, HDFS, Apache HBase, Apache Cassandra etc.
This step would consist of first introducing all of the selected technologies and going through how do they work, what are they supposed to solve and what are the advantages and disadvantages of using one.
After this, the section would do a comparison of these technologies in the context of stock data and conclude with analysis on which of the technologies would be the most prominent ones to solve this problem.

\subsection*{Step 3}

After this would start the experimental part of the thesis.
Based on the results of the step 2, I would implement couple of the most prominent solutions as a prototype systems.
This step would include the actual implementations and reporting of these implementations.
The report would consist of technical details; what parts does each of the system consists of, what versions were used, where the system was run etc. And subjective remarks; was it easy to implement, was there parts that did not fit together etc. 
The reporting part would also explain the metrics that would be measured for the subsequent analysis.
For these metrics, the dataset used to test the system would be the open-source REST API from IEX \cite{iex}, which is open for consumers until first of June, 2019.
These metrics could be, for example, the time it takes to batch process, processor usage, memory usage, database fetching times etc. 
As these systems would be run inside Docker containers, the tools used to measure these metrics would most probably be programming language specific functions and Docker specific statistics tools. 

\subsection*{Step 4}

In the final step, the thesis would first inspect the results from the step 3 and based on these make remarks on what could be the best potentially the best implementation in this context.
After this there would be an wrap-up on each of the previous sections concluding in retrospective what could've been possibly done better and what could be done in the future, concluding in recommendation what could be based on this thesis the best technical solution to implement system described in the application scenario. 

% for screenshots
%\begin{figure}[h]
%    \includegraphics[scale=0.5]{h2} 
%    \centering
%    \caption{}
%\end{figure}

\printbibliography

\end{document}